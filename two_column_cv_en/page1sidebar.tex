%!TEX root=mmayer.tex

\cvsection{Projects \& competitions}

\cvevent{\href{https://www.kaggle.com/c/liverpool-ion-switching}{\underline{Liverpool Ion Switching}} kaggle competition}{{2nd place (out of 2500+, top 1\%)}}{Mar 20 -- Jun 20}{}
\begin{itemize}
	\item Worked with time series/signal data
	\item Proposed winning solution for identifying the number of ion channels open at each time point
	\item \href{https://www.kaggle.com/c/liverpool-ion-switching/discussion/153991}{\underline{Blogpost.}} \href{https://github.com/stdereka/liverpool-ion-switching}{\underline{Full solution GitHub repository}}
\end{itemize}

\divider

\cvevent{Adaptive Importance Sampling in Monte-Carlo Population}{}{Mar 20 -- Jun 20}{}
\begin{itemize}
	\item Implemented an algorithm for variance reduction in Monte-Carlo integral estimation
	\item \href{https://github.com/stdereka/MAP361P}{\underline{GitHub repository}}
\end{itemize}

\cvsection{Skills \& Abilities}
\begin{itemize}
	\item Programming Languages: \\ 
	Python, C++, Java
	\item Linux terminal (Bash)
%	\item Basic routine chemical methods: titration, pH measurement; electrical signal observing methods (CRO/DSO).
	\item Python scientific stack: 
	numpy, matplotlib, scipy, pandas, jupyter-notebook. Several projects are on GitHub: \href{https://github.com/stdereka/calc\_math\_hometask}{\underline{numerical methods implementation},} \href{https://github.com/stdereka/PRM\_MIPT}{\underline{physical experiment data processing}}
	\item ML packages and deep learning frameworks. Sklearn, Keras, MXNet, Pytorch, Xgboost, Lasagne,
	Gensim
	\item SQL. Basic knowledge
	\item \LaTeX~system. Several projects are hosted on GitHub: \href{https://github.com/stdereka/yavor\_curse}{\underline{coursework report}}
	\item Frontend development. HTML, CSS, JavaScript stack. Basic knowledge
\end{itemize}


\cvsection{Courses}

\begin{itemize}
\item <<Machine Learning and Data Analysis>> specialization on \href{https://www.coursera.org/specializations/machine-learning-data-analysis?}{\underline{Coursera}} (MIPT and Yandex):
\begin{enumerate}
	\item \href{https://www.coursera.org/learn/supervised-learning?specialization=machine-learning-data-analysis}{<<Supervised learning>>}. Linear models, decision trees, random forest, ensembling, xgboost
	\item \href{https://www.coursera.org/learn/unsupervised-learning?specialization=machine-learning-data-analysis}{<<Unsupervised learning>>}. Cluster analysis, PCA
	\item \href{https://www.coursera.org/learn/stats-for-data-analysis/}{<<Statistics for data analysis>>}
	\item \href{https://www.coursera.org/learn/data-analysis-applications/}{<<Data analysis applications>>}. Time series, basics of NLP
	%\smallskip
\end{enumerate}


\divider
\item \href{https://en.dlschool.org/}{<<Deep Learning School>>} from \href{https::/mipt.ru/en}{\underline{MIPT}}. Computer vision, CNN, GAN
\item <<Basics of HTML and CSS>> on \href{https://www.coursera.org/learn/snovy-html-i-css?specialization=razrabotka-interfeysov}{\underline{Coursera}} (MIPT and Yandex)
\item <<Basics of JavaScript>> on \href{https://www.coursera.org/learn/javascript-osnovy-i-funktsii}{\underline{Coursera}} (MIPT and Yandex)

\end{itemize}

% \cvachievement{\faTrophy}{}{Received accolades at Atos for Best Performance in team.}
% \cvachievement{\faTrophy}{}{Received Best Debut Award at Atos. }
% %\divider
% \cvachievement{\faInstitution}{}{Won 2nd Consolation Prize for paper presented on Cognitive Radio Networks.}
% %\divider
% \cvachievement{\faGraduationCap}{}{Got Selected in "Exclusive Scholar Program" during undergrad.}
% %\divider
% \cvachievement{\faDollar}{}{Awarded with Narotam Sekhsaria Foundation Scholarship}
%\cvsection{Strengths}

%\cvtag{Hard-working (18/24)} 
%\cvtag{Persuasive}
%\cvtag{Motivator \& Leader}

%\divider\smallskip

%\cvtag{UX}
%\cvtag{Mobile Devices \& Applications}
%\cvtag{Product Management \& Marketing}


%\divider

%\cvevent{B.S.\ in Symbolic Systems}{Stanford University}{Sept 1993 -- June 1997}{}

%\cvsection{PUBLICATIONS}
%\smallskip
%\begin{itemize}
%\item R. Jain, H. Tulsani, A. Bansal, "A two- tier steganographic model based on (2,2)VCS and integer wavelet transform", in Preceding of The 5th International Conference on Computing for Sustainable Global Development organized by IEEE, pp. 4731-4734, (2018).
%\smallskip
%\end{itemize}
